\documentclass[12pt, letterpaper]{article}
\usepackage[utf8]{inputenc}
\usepackage[left = 2.5cm, right = 2.5cm, top = 2.5cm, bottom = 2.5cm]{geometry}
\usepackage{amsthm}
\usepackage{amsfonts}
\usepackage{amsmath}
\usepackage{amssymb}
\usepackage{graphicx}
\usepackage[T1]{fontenc}
\usepackage{listings}
\usepackage{hyperref}
\graphicspath{{images/}}

\author{Hernández Ferreiro Enrique Ehecatl \\
        López Soto Ramses Antonio \\
        Miguel Torres Eric Giovanni \\
        Quintero Villeda Erik}

\title{Práctica 9 \\
       {\small Fudamentos de Bases de Datos}}

\date{4 de noviembre de 2019}

\begin{document}
    \maketitle

    \section*{Introducción}
        
        \subsection*{Objetivo}
        Hacer uso de la sintaxis de consultas en T-SQL para obtener la solución requerida

    \section*{Desarrollo}
    Las consultas realizadas en T-SQL son las siguientes:

    \begin{itemize}
        \item[1.] El nombre completo con titulo de cortesía y cantidad de años laborando del 
                  empleado con más antigüedad.

                  \begin{lstlisting}
SELECT TOP(1) CONCAT(nombre, ' ', apellido) nombreCompleto, 
tituloDeCortesia, 
(YEAR(GETDATE()) - YEAR(fechaContratacion)) aniosLaborados
FROM Empleados
ORDER BY fechaContratacion ASC;
                  \end{lstlisting}

        \item[2.] Id del pedido, fecha de entrega, fecha de envío de los pedidos que
                  se hayan realizado por un empleado perteneciente al pais 'UK' cuyo
                  destino es Mexico.

                  \begin{lstlisting}
SELECT idPedido, fechaEntrega, fechaEnvio
FROM Pedidos INNER JOIN Empleados
ON Pedidos.idEmpleado = Empleados.idEmpleado
WHERE pais = 'UK' AND paisDestinatario = 'Mexico';
                  \end{lstlisting}

        \item[3.] El nombre completo y la fecha de entrega de los clientes que hayan
                  solicitado pedidos de productos con la categoría 'Seafood'.

                  \begin{lstlisting}
SELECT DISTINCT nombreContacto, fechaEntrega
FROM Clientes INNER JOIN Pedidos
ON Clientes.idCliente = Pedidos.idCliente
INNER JOIN DetallesPedido 
ON Pedidos.idPedido = DetallesPedido.idPedido
INNER JOIN Productos 
ON DetallesPedido.idProducto = Productos.idProducto
INNER JOIN Categorias 
ON Productos.idCategoria = Categorias.idCategoria
WHERE nombreCategoria = 'Seafood';
                  \end{lstlisting}

        \item[4.] Supón que la fecha actual es 1 de Enero de 1998. Encontrar los pedidos
                  de los productos descontinuados que aún no han llegado (se
                  considera que un producto no ha llegado si la fecha de entrega del
                  pedido es mayor a la actual o null).

                  \begin{lstlisting}
SELECT Pedidos.idPedido Pedido
FROM Pedidos INNER JOIN DetallesPedido
ON Pedidos.idPedido = DetallesPedido.idPedido
INNER JOIN Productos
ON DetallesPedido.idProducto = Productos.idProducto
WHERE descontinuado = 1 AND (fechaEntrega > '01-01-1998' OR fechaEntrega = NULL);
                  \end{lstlisting}

        \item[5.] Todos los productos que hayan sido despachados por los empleados
                  de la región 'Eastern'.

                  \begin{lstlisting}
SELECT DISTINCT nombreProducto
FROM Region INNER JOIN Territorios
ON Region.idRegion = Territorios.idRegion
INNER JOIN TerritoriosEmpleado
ON Territorios.idTerritorio = TerritoriosEmpleado.idTerritorio
INNER JOIN Empleados 
ON TerritoriosEmpleado.idEmpleado = Empleados.idEmpleado
INNER JOIN Pedidos
ON Empleados.idEmpleado = Pedidos.idEmpleado
INNER JOIN DetallesPedido 
ON Pedidos.idPedido = DetallesPedido.idPedido
INNER JOIN Productos
ON DetallesPedido.idProducto = Productos.idProducto
WHERE descripcionRegion = 'Eastern';
                  \end{lstlisting}

    \end{itemize}

    \section*{Conclusión}
    Como se puede notar el objetivo de la práctica fue alcanzado. Las consultas resultaron ser más
    sencillas de lo que parecían.\vspace{.3cm}

    Sólo tuvimos pocos problemas al momento de implementar las consultas, pues al tener que unir
    varias tablas, nos confundíamos al momento de poner las condiciones para poder juntarlas, pero 
    no fue un problema de mucha relevancia.\vspace{.3cm}

    En resumen, las consultas fueron resueltas de manera exitosa.

\end{document}